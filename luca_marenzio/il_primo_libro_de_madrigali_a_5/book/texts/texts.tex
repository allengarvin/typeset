\documentclass[12pt, twocolumn]{text-translation}

\begin{document}

% Title, centered for the whole page
\clearpage
\onecolumn
\vspace*{\fill}
\begin{center}
    \Huge Texts, Translations and Commentaries
\end{center}
\vspace*{\fill}

% Transition to two-column format
\twocolumn

\noindentation
The texts of each madrigal are presented in this section with translations
by the editor. These translations are not meant to be song, nor are they
aimed at presenting a true literary representation of the themes of the
original text. Instead, they are presented here to aid singers or players
with interpreting the text, attempting a fairly literal translation, and
where possible, within the bounds of the English language, to represent
faithfulness to original Italian on a line-by-line basis. Occasionally to
avoid torturing the syntax too much, this has not been possible.

\indented
Cinquecento madrigal prints frequently offer no punctuation apart from a final
full stop. Typically, all other punctuation has been supplied by the editor
according to his sense of the meaning of the poem. Alternative readings are
possible, and singers should feel free to develop alternative phrasings
informed by their own knowledge of Italian and their opinions on the text
in question.

\indented
The Critical Commentary in Appendex I (forthcoming) will give the original text, with
original spelling and any existing punctuation, plus differences or errors
between the partbooks.

% Poem 1
\poemtitle{Liquide perle, Amor}

Liquide perle, Amor dagli occhi sparse, \\
in premio del mio ardore, \\
ma, lasso, ohime! che'l core \\
di maggior foco m'arse; \\
Ahi, che bastava solo \\
a darmi morte il primo ardente duolo.
\poemasterisks
Liquid pearls from Love's eyes scattered \\
in reward for my ardor, \\
but, ah, alas! my heart \\
burns from a fiercer flame; \\
Ah, it alone sufficed, \\
that first ardent pain, to bring me death. 
% translation 2023-11-11

\poemsection{Poetic form}
Madrigale, with rhyme scheme AbbacC.

\poemsection{Commentary}
This piece metaphorically likens domestic cats to their wilder cousins, emphasizing their shared grace and mystery.

% Poem 2
\poemtitle{Ohimè! dov'è il mio ben}
Ohimè! dov'è il mio ben, dov'è il mio core?  \\
Chi m'asconde il mio core, e chi me'l toglie?  \\
Dunque ha potuto sol desio d'onore  \\
darmi fera cagion di tante doglie?  \\
Dunque ha potuto in me più che il mio amore,  \\
ambitiose e troppo lievi voglie?  \\
Ahi sciocco mondo e cieco! Ahi cruda sorte,  \\
che ministro mi fai della mia morte!  

{\raggedleft \textit{Bernardo Tasso (1493-1569)}\par}

\poemasterisks
Alas! Where is beloved, where is my heart? \\
Who has hidden my own heart from me, and who takes it from me? \\
Could the desire alone for honor \\
give me such cruel cause for so much pain? \\
Could ambitious and too fleeting desires \\
have meant more to me that my love? \\
Ah foolish and blind world! Ah, cruel fate, \\
that renders me the minister of my own death!
% translation 2023-11-11

\poemsection{Poetic form}
Ottava rima, with the standard ABABABCC scheme.

\poemtitle {Spuntavan già (prima parte)}
Spuntavan già per far il mondo adorno \\
vaghi fioretti, erbette, verdi e belle \\
di color mille e'n queste parti e'n quelle \\
rallegravan la terra e i colli intorno. \\
Gian gli augelletti all' apparir del giorno \\
d'amor cantando sin sovra le stelle \\
e correvan le fiere ardite e snelle \\
tra lor scherzando, alle campagne intorno. \\

\poemasterisks
Emerging already to adorn the world were \\
happy little flowers, grasses green and lovely, \\
of a thousand colors, hither and yon, \\
to gladden the earth and the surrounding hills. \\
Already the little birds at the appearance of day, \\
singing of love to the fading stars above \\
and dashing about the bold and quick beasts, \\
sporting amongst themselves, in the countryside around. \\

\poemsection{Poetic form}
Octet of sonnet (ABBAABBA).

\poemtitle {Quando'l mio vivo sol (seconda parte)}
Quando'l mio vivo sol perch'io non pera \\
godi or, mi disse con un dolce riso: \\
amante fido il premio del tuo ardore. \\
Indi con molti bacci sparse fuore \\
quante grazie e dolcezze ha'l Paradiso \\
e quant'a odor nei fior la Primavera. \\

\poemasterisks
When my living sun, so that I not perish, \\
Enjoy now (she said to me with a sweet smile) \\
my faithful lover, the reward for your passion. \\
Then with many kisses she spread forth  \\
all the graces and sweetnesses Paradise has \\
and all the fragrances of flowers of the Spring. \\

\poemsection{Poetic form}
Sestet of sonnet (ABCCBA).

\poemtitle {Quando i vostri begli occhi}
Quando i vostri begli occhi un caro velo  \\
ombrando copre semplicetto e bianco,  \\
d'una gelata fiamma il cor s'alluma,  \\
madonna, e le midolle un caldo gelo  \\
trascorre sì, ch'a poco a poco io manco,  \\
e l'alma per diletto si consuma:   \\
Così morendo vivo; e con quell' arme  \\
che m'uccidete, voi potete aitarme.  

{\raggedleft \textit{Jacopo Sannazaro (1458-1530)}\par}
\poemasterisks

When your lovely eyes are shaded by an intimate veil, \\
a covering simple and white, \\
my heart is kindled by a frozen flame, \\
my lady, and in my bones a hot chill  \\
passes through, such that little by little, \\
in delight my soul is consumed: \\
Thus dying, I live, and with those arms \\
by which you slew me, you may sustain me.  \\

\poemsection{Poetic form}
Canzone with rhyme ABCABCDD.

\poemtitle {Tirsi morir volea}
Tirsi morir volea, \\
gli occhi mirando di colei ch'adora \\
ond' ella che di lui non meno ardea, \\
gli disse: Ohimè ben mio, \\
Deh non morir ancora! \\
che teco bramo di morir anch'io! \\
 \\
Frenò Tirsi il desio, \\
Ch'avea di pur sua vita allor finire; \\
e sentea morte, e non potea morire. \\
E mentre fisso il guardo pur tenea \\
ne begli occhi divini \\
la bella Ninfa sua, che già vicini \\
sentea i messi d'Amore, \\
disse, con occhi languidi e tremanti: \\
Mori, cor mio, ch'io moro! \\
Le rispose il Pastore: \\
Ed io, mia vita, moro! \\
 \\
Così morirò i fortunati amanti \\
di morte sì soave e sì gradita, \\
che per anco morir tornaro in vita. 

{\raggedleft \textit{Battista Guarini (1538-1612)}\par}
\poemasterisks
Tirsi wished to die, \\
gazing into the eyes of she whom he adored, \\
whereupon she, who burned no less for him, \\
told him: Alas, my beloved, \\
ah, die not yet! \\
For with you I too yearn to die! \\
 \\
Tirsi restrained his desire \\
that he had to thus end his life right then; \\
and he felt death, and he could not die. \\
And while he held his gaze fixed \\
upon the alluring divine eyes of \\
his beautiful nymph, who already felt near \\
the messengers of Love; \\
she said, with languid and trembling eyes: \\
Die, my heart, for I am dying, \\
The shepherd responded to her: \\
And I, my life, die! \\
 \\
Thus died the fortunate lovers, \\
of a death so sweet and so welcome \\
that they returned to life, to die yet again. 

\poemsection{Poetic form}
Madrigal in 3 sections (aBAcbC, aBBcdDeFgeg, ABB).

\poemtitle {Dolorosi martir}
Dolorosi martir, fieri tormenti,  \\
duri ceppi, empi lacci, aspre catene,  \\
ov'io la notte, i giorni, ore e momenti  \\
misero piango il mio perduto bene;  \\
Triste voci, querele, urli, e lamenti,  \\
lagrime spesse e sempiterne pene  \\
son' il mio cibo e la quiete cara  \\
della mia vita, oltre ogni assenzio amara. 

{\raggedleft \textit{Luigi Tansillo (1510-1568)}\par}
\poemasterisks
Sorrowful sufferings, fierce torments,  \\
harsh shackles, cruel snares, savage chains,  \\
where I, by night, by day, by hour and moment  \\
miserably weep for my lost love;  \\
Sad voices, complaints, howls and laments,  \\
frequent tears and unending pains  \\
are my nourishment and the precious peace  \\
of my life is now bitter as wormwood. 

\poemsection{Poetic form}
Ottava rima, with the standard ABABABCC scheme.

\poemtitle {Che fa oggi il mio sole}
Che fa oggi il mio sole? \\
che fa'l mio canto e'l suono, \\
che non cantan di lei, la gloria e'l nome? \\
Or queste mie viole \\
e questi fior gli dono,  \\
che ne facci corona alle sue chiome. 
\poemasterisks
What does my sun do today? \\
What do my song and my sound do, \\
if they do not sing of her, her glory, and her name? \\
Now these violets of mine \\
and these flowers I give to her, \\
that she make a crown of them upon her hair.

\poemsection{Poetic form}
Madrigal with rhyme scheme abCabC.

\poemtitle {Lasso ch'io ardo}
Lasso ch'io ardo e'l mio bel sole ardente  \\
i suoi bei raggi d'oro  \\
volge in altr'oriente,  \\
ivi imperla, ivi indora ed io mi moro.   \\
Amor, deh torna a me torna la chiara  \\
bella mia luce e cara. 
\poemasterisks
Alas, for I burn, and my beautiful blazing sun  \\
turns her alluring golden rays  \\
in another direction,  \\
there covered in pearls, there gilded, and I die.  \\
Love, ah, return to me, return that bright  \\
beautiful light of mine, so dear.  \\

\poemsection{Poetic form}
Madrigal with rhyme scheme AbaBCc.

\poemtitle {Venuta era Madonna (prima parte)}
Venuta era Madonna al mio languire  \\
con dolce aspetto umano, \\
allegra e bella in sonno a consolarme; \\
ed io, prendendo ardire, \\
di dirle quanti affanni ho speso in vano, \\
vidila con pietade a sé chiamarme,  \\
dicendo: A che sospire,  \\
a che ti struggi ed ardi di lontano? \\
Non sai tu che quell'arme \\
che fer la piaga ponno il duol finire?

{\raggedleft \textit{Jacopo Sannazaro (1458-1530)}\par}
\poemasterisks
My Lady has come to me in my languishing  \\
with sweet, humane countenance, \\
happy and lovely in my sleep to console me;  \\
and I, boldly venturing  \\
to tell her how many afflications I've spent in vain, \\
witnessed her with pity call me to her \\
saying: why do you sigh, \\
why do torment yourself and burn from afar? \\
Do you not know that these arms [weapons] \\
that struck the wound can end your suffering? 

\poemsection{Poetic form}
Canzone with scheme AbCaBCaBcA.

\poemtitle {Intanto il sonno (seconda parte)}
Intanto il sonno si partia pian piano,  \\
ond'io per ingannarme  \\
lungo spazio non volsi gli occhi aprire,  \\
ma dalla bianca mano  \\
che si stretta tenea sentii lasciarme. 

{\raggedleft \textit{Jacopo Sannazaro (1458-1530)}\par}
\poemasterisks
While my sleep departed slowly,  \\
so that to deceive myself  \\
for a long space I did not want to open my eyes,  \\
but by her white hand  \\
that was clasped tightly [to mine], I felt her release me.  \\

\poemsection{Poetic form}
Canzone continuing prima parte with BcAbC.

\poemtitle {Madonna mia gentil}
Madonna mia gentil, ringrazio Amore  \\
che tolto m'abbia il core \\
dandolo a voi ch'avete \\
non sol beltà ma sete  \\
ornata di virtù tal che m'avviso \\
stando in terra godere il Paradiso.
\poemasterisks
My gentle lady, I thank Love \\
who stole away my heart, \\
giving it to you, who has \\
not only beauty but is \\
adorned with such virtue that I believe \\
while remaining on earth, enjoys Paradise. 
\poemsection{Poetic form}
Madrigal with rhyme AabbCC.

\poemtitle {Cantava la più vaga pastorella}
Cantava la più vaga pastorella   \\
che mai premesse fiori   \\
e scopriva nel viso almi colori,  \\
una ninfa di lei molto più bella.  \\
Deh, perché l'alma, fatta ad ambe ancella   \\
non ebbe allor duo cori,   \\
mentr'era a l'un e all'altra intento e fisso   \\
per lassarne uno al canto e l'altro al viso? 
\poemasterisks
The fairest shepherdess  \\
who ever tripped across flowers sang,  \\
and a nymph of yet greater beauty  \\
revealed the noble colors of her face.  \\
Ah, why di my soul, made servant to both,  \\
not then have two hearts,  \\
while it was upon one and the other intent and fixed  \\
to leave one for the song and the other for the face? 
\poemsection{Poetic form}
Madrigal with rhyme scheme AbBAAbCC.

\poemtitle {Questa di verdi erbette}
Questa di verdi erbette  \\
e di novelli fior tessuta or ora  \\
vaga e gentil ghirlanda,  \\
giovin pastor, ti manda  \\
l'amata e bella Flora,  \\
che con le sue caprette  \\
sta in riva al Tebro soggiornando e dice,  \\
ch'ivi or t'aspetta e ti vo' far felice. 
\poemasterisks
Just now woven of green herbs  \\
and of fresh flowers,  \\
this lovely and dainty garland,  \\
is sent to you, O youthful shepherd,  \\
by your beloved and beautiful Flora,  \\
who with her young goats,  \\
tarries on the banks of the Tiber and says  \\
that she now awaits you and wishes to make you happy.

\poemsection{Poetic form}
Madrigal with rhyme aBccbaDD.

\poemtitle {Partirò dunque}
Partirò dunque, ohimè mi manca il core,  \\
porgimi aita Amore! \\
Come esser può ch'io viva \\
lontan da quel bel sguardo \\
per cui sì com'or ardo \\
con estremo dolore? \\
Allor via più sentiva  \\
maggior dolcezza quanto più maggiore \\
era quel vivo ardore;  \\
Prestami aiuto, Amore. \\
\poemasterisks
I shall thus leave, alas! my heart fails me, \\
bring me aid, Love! \\
How can it be that I live \\
far from that beautiful gaze \\
for which I now burn \\
with extreme torment?  \\
The more sweetness that was felt then, \\
the greater the living ardor was; \\
Lend me your aid, Love.  \\

\poemsection{Poetic form}
Madrigal with rhyme AabccabAaa.

\poemtitle {O tu che fra le selve}
O tu che fra le selve occulta vivi, \\
ch'è della vita mia? ch'è del mio Amore? \\
\hspace*{1cm}\textit{More} \\
Dunque, Ninfa gentil, se lei si more, \\
non vedrò le sue luci a fé giammai? \\
\hspace*{1cm}\textit{Mai} \\
Che farò dunque in sì noiosa vita? \\
Chi mi sonsolerà nel stato mio? \\
\hspace*{1cm}\textit{Io} \\
E tu, come ti chiami, miserella, \\
che mi consola voi in questo speco? \\
\hspace*{1cm}\textit{Eco} \\
Ecco gentil che negli ultimi accenti \\
mi risponde, non son d'amanti esempio? \\
\hspace*{1cm}\textit{Empio} \\
E perché mi risponde ch'io son empio? \\
Non ho avuto pietà di suoi lamenti? \\
\hspace*{1cm}\textit{Menti} \\
Mentir non posso che'l ciel e le stelle \\
ponno far fede s'io gl'ho dato guai! \\
\hspace*{1cm}\textit{Hai} \\
Or sia come si voglia, addio, ti lasso, \\
spirto c'hai voce e fra gli boschi vivi, \\
or quanto ho detto fra gli tronchi scrivi.

{\raggedleft \textit{Attr. Torquato Tasso (1544-1595)}\par}
\poemasterisks
You who who live hidden amidst the woods, \\
what of my life? What of my love? \\
\hspace*{1cm}\textit{She dies} \\
Thus, gentil nymph, if she dies \\
shall I truly never again see her eyes [lights]? \\
\hspace*{1cm}\textit{Never} \\
What then shall I do, in my so wretched life? \\
Who will console me in my state? \\
\hspace*{1cm}\textit{I} \\
And you, what are you called, o miserable one, \\
who consoles me in this valley? \\
\hspace*{1cm}\textit{Echo} \\
Behold the gentle one that in her last words \\
replies to me, am I not an example for lovers? \\
\hspace*{1cm}\textit{Cruel} \\
And why do you reply to me that I'm cruel? \\
Did you have no pity for her laments? \\
\hspace*{1cm}\textit{You lie} \\
I cannot lie, for the heavens and the starts \\
may bear witness if I have given her woes! \\
\hspace*{1cm}\textit{You have} \\
Now, let is be as it may, adieu, I leave you, \\
spirit, who has voice and dwells among the trees, \\
now carve whatever I've said on these tree trunks. \\


\poemsection{Commentary} This is an example of an echo poem,
a comedic device in both where the conceit is that an echo
produces the final syllables of the last line spoken by the
speaker are repeated, and misunderstood by the speaker. It was used
in the Commedia dell'Arte as well as in more elevated verse, and
examples exist in 16th century French and English as well. The
19th century attribution to Tasso is considered doubtful.

\end{document}

