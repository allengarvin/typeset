\documentclass{article}
\usepackage{booktabs}
\usepackage{tabularx}
\usepackage[margin=1in]{geometry}
\usepackage{caption}
\usepackage{lmodern}
\usepackage{array} % For better column formatting
\usepackage{fontspec}
\newfontfamily\musicsymbols{Bravura} 

\newcommand{\commonTime}{\raisebox{0.9ex}{{\musicsymbols\Large\symbol{"1D134}}}}
\newcommand{\cutTime}{\raisebox{0.9ex}{{\musicsymbols\Large\symbol{"1D135}}}}
\newcommand{\commonTimeThreeTwo}{%
  \raisebox{0.9ex}{{\musicsymbols\Large\symbol{"1D134}}}%
  \hspace{0.3em}%
  \raisebox{0.0ex}{\shortstack[c]{\scriptsize\textbf{3}\\\scriptsize\textbf{2}}}%
}

% Define a helper column type for multi-line headers
\newcolumntype{L}[1]{>{\raggedright\arraybackslash}p{#1}}

\begin{document}

\vspace{1em}

% Centered Table Title with Large First Letter
\noindent\centering
\textsc{\Large T}ABLE 1\\[0.5ex]
Overall layout for \textit{Il primo libro de' madrigali a cinque voci}
\par

\vspace{0.5ex}
\noindent\rule{\linewidth}{0.4pt}

% Table with fixed column widths and multi-line headers
\begin{tabularx}{\textwidth}{@{}L{0.9cm} L{4.2cm} L{1cm} L{1.0cm} L{1cm} L{2.4cm} L{2.2cm} L{2cm}@{}}
\textbf{No.} &
\textbf{Cleffing} &
\textbf{Key} \newline \textbf{sig} &
\textbf{Final} &
\textbf{Meter} &
\textbf{Length} \newline \textbf{in breves}\textsuperscript{a} &
\textbf{Number of} \newline \textbf{poetic lines}\textsuperscript{b} &
\textbf{Poetic} \newline \textbf{Form} \\
\midrule
1 & G2,G2,C2,C3,F3 & \textemdash & G & \commonTime & 29 & 6 & madrigal \\
2 & G2,C2,C3,C3,F3 & \textemdash & C & \commonTime & 37 & 8 & ottava rima \\
3-4 & G2,G2,C2,C3,F3 & \textemdash & D & \commonTime & 74 (36+38) & 14 (8+6) & sonnet \\
5 & G2,C2,C3,C3,C4 & \textemdash & A & \commonTime & 38 & 8 & ottava rima \\
6-8 & C1,C3,C4,C4,F4 & \textemdash & E & \commonTime & 75 (29+30+16) & 20 & madrigal \\
9 & C1,C3,C4,C4,F4 & \textemdash & E & \cutTime & 63 & 8 & ottava rima \\
10 & C1,C1,C3,C4,F4 & \textemdash & D & \commonTime & 26 & 6 & madrigal \\
11 & C1,C3,C4,C4,F4 & \textemdash & A & \commonTime & 30 & 6 & madrigal \\
12-13 & G2,C2,C3,C3,F3 & \textemdash & C & \cutTime & 77 (41+36) & 15 (10+5) & canzone \\
14 & G2,C1,C2,C3,C4 & $\flat$ & G & \commonTime & 27 & 6 & madrigal \\
15 & G2,G2,C2,C3,F3 & $\flat$ & G & \commonTime & 37 & 8 & madrigal \\
16 & C1,C1,C3,C4,F4 & $\flat$ & G & \commonTime & 45 & 8 & madrigal \\
17 & C1,C1,C3,C4,F4 & $\flat$ & G & $\commonTimeThreeTwo$ & 36 & 10 & madrigal \\
18 & C1,C1,C3,C3,C4,C4,F4,F4 & $\flat$ & G & \commonTime & 64 & 17 & villanella \\
\bottomrule
\end{tabularx}

\vspace{1em}

% Footnotes
\raggedright
\noindent\textit{a.} With multipart madrigals, the total length is given first followed by the length of each \textit{parte}.\\
\noindent\textit{b.} For multipart madrigals, the total lines is given first followed by the no. of poetic lines in each \textit{parte}. 

\vspace{0.5em}
\noindent\rule{\textwidth}{0.4pt}

\end{document}
